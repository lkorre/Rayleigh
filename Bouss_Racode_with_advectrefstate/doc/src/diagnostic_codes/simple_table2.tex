\documentclass[10pt, letterpaper]{article}
\usepackage[letterpaper,margin=0.75in]{geometry}
\usepackage{graphicx}
\usepackage{amsmath}
\usepackage[table]{xcolor}
%\usepackage[thinlines]{easytable}
\DeclareMathSizes{10}{10}{10}{10}
\begin{document}

\title{Introduction to \LaTeX{}}
\author{Author's Name}

\maketitle

\begin{abstract}
The abstract text goes here.
\end{abstract}

\section{Velocity Field \& Related Variables}
The velocity field is defined 

\begin{equation}
    \label{simple_equation}
    \alpha = \sqrt{ \beta }
\end{equation}

\subsection{Subsection Heading Here}
Write your subsection text here. here is a much longer line of text to see if I can understand where the margins land.


\rowcolors{2}{gray!25}{white}
\begin{table}
\centering
\begin{tabular}{|ccc||ccc||ccc|}
\hline
Expression & Code & Variable & Expression & Code & Variable & Expression & Code & Variable \\
\hline
\input velocity_field_table0.tex
\hline
\end{tabular}
\end{table}

\rowcolors{2}{gray!25}{white}
\begin{table}
\centering
\begin{tabular}{|ccc||ccc||ccc|}
\hline
Expression & Code & Variable & Expression & Code & Variable & Expression & Code & Variable \\
\hline
\input velocity_field_table1.tex
\hline
\end{tabular}
\end{table}

\newpage

\subsection{Mass Flux}
\rowcolors{2}{gray!25}{white}
\begin{table}
\centering
\begin{tabular}{|ccc||ccc||ccc|}
\hline
Expression & Code & Variable & Expression & Code & Variable & Expression & Code & Variable \\
\hline
\input mass_flux_table0.tex
\hline
\end{tabular}
\end{table}
\subsection{Vorticity}
\rowcolors{2}{gray!25}{white}
\begin{table}
\centering
\begin{tabular}{|ccc||ccc||ccc|}
\hline
Expression & Code & Variable & Expression & Code & Variable & Expression & Code & Variable \\
\hline
\input vorticity_field_table0.tex
\hline
\end{tabular}
\end{table}
\section{Conclusion}
Write your conclusion here.

\end{document}
