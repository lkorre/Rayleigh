\section{Diagnostic Outputs}\label{sec:diagnostics}

Rayleigh comes bundled with an in-situ diagnostics package that allows the user to sample a simulation in a variety of ways, and at user-specified intervals throughout a run.  This package is comprised of roughly 17,000 lines of code (about half of the Rayleigh code base), and it is complex enough that we describe it in two other documents.  We refer the user to :
\begin{enumerate}
\item The diagnostics plotting manual, provided in three formats:
    \begin{itemize}
    \item Rayleigh/etc/analysis/Diagnostics\_Plotting.ipynb (Jupyter Python notebook format; recommended for interactive use)
    \item Rayleigh/doc/Diagnostics\_Plotting.html  (recommended for optimal viewing; generated from the .ipynb file)
    \item Rayleigh/doc/Diagnostics\_Plotting.pdf   (same content as .html and .ipynb, but formatting quality is inferior)
    \end{itemize}
\item Rayleigh/doc/rayleigh\_output\_variables.pdf --  This companion document provides the output menu system referred to in the main diagnostics documentation.
\end{enumerate}

A number of stand-alone Python plotting examples may also be found in the Rayleigh/etc/analysis directory.
