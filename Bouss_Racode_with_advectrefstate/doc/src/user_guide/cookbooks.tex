\clearpage


\section{Cookbooks}\label{sec:cookbooks}
\emph{This section has been written by Maria Weber.}

In this section, we explain and document example Rayleigh main input files for a variety of problems. The example input files can be found in directory \textbf{Rayleigh/input\_examples/}. In some cases, we will also show example diagnostic outputs. See \S \ref{sec:diagnostics} for more information on generating Rayleigh diagnostic routines with Python. 

Standard benchmarks that generate minimal output files are discussed in \S \ref{sec:cookbook_case0_minimal}-\ref{sec:cookbook_mhd_anelastic}. 


\subsection{Simple Boussinesq non-MHD benchmark: c2001\_case0\_minimal}\label{sec:cookbook_case0_minimal}

This is the standard benchmark test when running Rayleigh on a new machine, as described in \S \ref{sec:benchmarking}. Christensen et al. (2001) describes two Boussinesq tests that Rayleigh's results may be compared against. Case 0 in Christensen et al. (2001) solves for Boussinesq (non-dimensional) non-magnetic convection, and we will discuss the input parameters necessary to set up this benchmark in Rayleigh below. Rayleigh's input parameters are grouped in so-called namelists, which are subcategories of related input parameters that will be read upon program start and assigned to Fortran variables with identical names. Below are the first four Fortran namelists in the input file \textbf{c2001\_case0\_minimal}.
    
\begin{lstlisting}
&problemsize_namelist
 n_r = 64
 n_theta = 96
 nprow = 16
 npcol = 32
/
&numerical_controls_namelist
/
&physical_controls_namelist
 benchmark_mode = 1
 benchmark_integration_interval = 100
 benchmark_report_interval = 5000
/
&temporal_controls_namelist
 max_iterations = 25000
 checkpoint_interval = 100000
 quicksave_interval = 10000
 num_quicksaves = 2
/
\end{lstlisting}    

In namelist \texttt{problemsize\_namelist}, the number of radial grid points is denoted by \texttt{n\_r} and the number of $\theta$ grid points by \texttt{n\_theta}. For optimal load-balancing, the number of MPI ranks distributed within a row is denoted by \texttt{nprow} and within a column is \texttt{npcol}. See \S \ref{sec:running} for instructions on appropriately defining these values.

When running a benchmark, set \texttt{benchmark\_mode} under \texttt{physical\_controls\_namelist} to the code number for the corresponding benchmark you want to run. When benchmark mode is active, custom inputs are overridden and reset to their benchmark appropriate value (see \S \ref{sec:benchmarking}). Setting \texttt{benchmark\_mode = 1} defines the appropriate Case 0 Christensen et al. (2001) initial conditions. A benchmark report is written every 5000 time steps  by setting \texttt{benchmark\_report\_interval = 5000}. The benchmark reports are text files found within directory \textbf{path\_to\_my\_sim/Benchmark\_Reports/} and numbered according to the appropriate time step. The \\\texttt{benchmark\_integration\_interval} variable sets the interval at which measurements are taken to calculate the values reported in the benchmark reports.    

Within \texttt{temporal\_controls\_namelist}, the maximum number of iterations is set with \texttt{max\_interations}. Checkpoints are written at time step intervals set by \texttt{checkpoint\_interval}. In this case, the checkpoint interval is larger than the maximum number of iterations, so no checkpoint will be written. The interval at which quicksaves are written is set by variable \texttt{quicksave\_interval} and the number of quicksaves saved on disk at a time is set by \texttt{num\_quicksaves}. See \S \ref{sec:quicksaves} for more information on quicksaves. 

Upon completion of this benchmark, verify that your installation is working correctly by comparing the file \\ \textbf{path\_to\_my\_sim/Benchmark\_Reports/00025000} to Table \ref{table:bench_high}. All values should have a percent difference of less than 1. 


\subsection{Simple Boussinesq MHD benchmark: c2001\_case1\_minimal}

The MHD Boussinesq benchmark with an insulating inner core of Christensen et al. (2001) is denoted as Case 1 and is specified with input file \textbf{c2001\_case1\_minimal}. Only the namelists modified compared to Case 0 (\S \ref{sec:cookbook_case0_minimal} above) are shown below.   
    
\begin{lstlisting}
&physical_controls_namelist
 benchmark_mode = 2
 benchmark_integration_interval = 100
 benchmark_report_interval = 10000
/
&temporal_controls_namelist
 max_iterations = 150000
 checkpoint_interval = 100000
 quicksave_interval = 10000
 num_quicksaves = 2
/
\end{lstlisting}    

In this example, \texttt{benchmark\_mode = 2} sets the benchmark-appropriate values for Christensen et al. (2001) Case 1. The variable \texttt{benchmark\_integration\_interval} remains the same as Case 0 above, but the \texttt{benchmark\_report\_interval} has been increased in this MHD problem. Here, \texttt{max\_iterations} has also been increased compared to Case 0 such that it is now larger than \texttt{checkpoint\_interval}. As such, checkpoint files for time step 100000 will be written in directory \textbf{path\_to\_my\_sim/Checkpoints/00100000}. Upon completion of this benchmark, verify that your installation is working correctly by looking at file \textbf{path\_to\_my\_sim/Benchmark\_Reports/00150000}. All values should have a percent difference of less than 1.   


\subsection{Steady anelastic non-MHD benchmark: j2011\_steady\_hydro\_minimal}\label{sec:cookbook_hydro_anelastic}

Jones et al. (2011) describes a benchmark for an anelastic hydrodynamic solution that is steady in a drifting frame. This benchmark is specified for Rayleigh with input file \textbf{j2011\_steady\_hydro\_minimal}. Below are the relevant Fortran namelists.     
    
\begin{lstlisting}
&problemsize_namelist
 n_r = 128
 n_theta = 192
 nprow = 32
 npcol = 16
/
&numerical_controls_namelist
/
&physical_controls_namelist
 benchmark_mode = 3
 benchmark_integration_interval = 100
 benchmark_report_interval = 10000
/
&temporal_controls_namelist
 max_iterations = 200000
 checkpoint_interval = 100000
 quicksave_interval = 10000
 num_quicksaves = 2
/
\end{lstlisting}    

Suggested problem size values are given in \texttt{problemsize\_namelist}, along with variables for \texttt{physical\_controls\_namelist} and \texttt{temporal\_controls\_namelist}. The variable \texttt{benchmark\_mode = 3} designates appropriate input conditions for the Jones et al. (2011) anelastic hydrodynamic benchmark. Upon completion of this benchmark, verify that your installation is working correctly by looking at file \textbf{path\_to\_my\_sim/Benchmark\_Reports/00200000}. All values should have a percent difference of less than 1. 


\subsection{Steady anelastic MHD benchmark: j2011\_steady\_mhd\_minimal}\label{sec:cookbook_mhd_anelastic}

The anelastic MHD benchmark described in Jones et al. (2011) can be run with main input file \textbf{j2011\_steady\_mhd\_minimal}. The Fortran namelists differing from the Jones et al. (2011) anelastic hydro benchmark (\S \ref{sec:cookbook_hydro_anelastic} above) are shown here.      
    
\begin{lstlisting}
&physical_controls_namelist
 benchmark_mode = 4
 benchmark_integration_interval = 100
 benchmark_report_interval = 10000
/
&temporal_controls_namelist
 max_iterations = 5000000
 checkpoint_interval = 100000
 quicksave_interval  = 25000
 num_quicksaves = 2
/
\end{lstlisting} 

You may wish to modify the problem size within \texttt{problemsize\_namelist} (particularly \texttt{nprow} and \texttt{npcol}), explained in more detail in \S \ref{sec:cookbook_case0_minimal}. The variable \texttt{benchmark\_mode = 4} designates appropriate input conditions for the Jones et al. (2011) anelastic MHD benchmark. Here, \texttt{max\_iterations} has also been increased compared to the anelastic hydro benchmark of Jones et al. (2011), as well as \texttt{quicksave\_interval}. Upon completion of this benchmark, verify that your installation is working correctly by looking at file \textbf{path\_to\_my\_sim/Benchmark\_Reports/05000000}. All values should have a percent difference of less than 1.    
  



 
    
    
    
    